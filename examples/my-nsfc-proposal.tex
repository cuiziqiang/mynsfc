%%% my-nsfc-proposal.tex ---
%%
%% Filename: my-nsfc-proposal.tex
%% Author: Fred Qi
%% Created: 2016-02-04 15:26:58(-0700)
%%
%% Last-Updated: 2018-07-01 16:00:28(+0800) [by Fred Qi]
%%     Update #: 74
%%%%%%%%%%%%%%%%%%%%%%%%%%%%%%%%%%%%%%%%%%%%%%%%%%%%%%%%%%%%%%%%%%%%%%
%%
%%% Commentary:
%%
%%
%%%%%%%%%%%%%%%%%%%%%%%%%%%%%%%%%%%%%%%%%%%%%%%%%%%%%%%%%%%%%%%%%%%%%%
%%
%%% Change Log:
%%
%%
%%%%%%%%%%%%%%%%%%%%%%%%%%%%%%%%%%%%%%%%%%%%%%%%%%%%%%%%%%%%%%%%%%%%%%

\documentclass[subfig,boldtoc]{mynsfc}

\bibliography{my-nsfc-proposal}

% 名称与申请人(不会在正文部分出现)
% \title{自然科学基金正文\XeLaTeX{}模板}
\title{报告正文}
\author{齐飞}

\begin{document}

% 不能编写页码
\thispagestyle{empty}

%% 大字体显示“报告正文”字样
\maketitle
% \begin{center}
%   \kaiti \erhao \bfseries 报告正文
% \end{center}

\part{立项依据与研究内容}
\label{part:proposal}

\section{项目的立项依据}
\label{sec:background}

\begin{hcomment}
  (研究意义、国内外研究现状及发展动态分析,需结合科学研究发展趋势来论述科学意
  义; 或结合国民经济和社会发展中迫切需要解决的关键科技问题来论述其应用前景。附主
  要参考文献目录)
\end{hcomment}

\subsection{研究意义}
\label{sec:motivation}

略。

\subsection{国内外研究现状}
\label{sec:review}

可以引用相关文献~\cite{bengio_representation_2013}对研究意义与国内外研究现状进行说明。

\printbibliography[heading=reftype,title={参考文献}]

\section{项目的研究内容、研究目标,以及拟解决的关键科学问题}
\label{sec:contents}

\begin{hcomment}
  (此部分为重点阐述内容)
\end{hcomment}

\subsection{研究内容}

略。

\subsection{研究目标}

略。

\subsection{拟解决的关键科学问题}

略。

\section{拟采取的研究方案及可行性分析}
\label{sec:approach}

\begin{hcomment}
  (包括研究方法、技术路线、实验手段、关键技术等说明)
\end{hcomment}

略。

\section{本项目的特色与创新之处}
\label{sec:innovation}

略。

\section{年度研究计划及预期研究结果}
\label{sec:plan}

\begin{hcomment}
  包括拟组织的重要学术交流活动、国际合作与交流计划等)
\end{hcomment}

略。

\part{研究基础与工作条件}
\label{sec:preparation}


\section{工作基础}
\label{sec:previous-work}

\begin{hcomment}
  (与本项目相关的研究工作积累和已取得的研究工作成绩)
\end{hcomment}


\begin{refsection}

  申请人针对\textbf{某问题}进行了研究~\cite{xia_saliency_2015}。下面命令
  \texttt{initauthors} 中的哈希字符串可以在Biber/bibtex 生成的文件
  \texttt{*.bbl} 中找到。

  需要强调的作者列表有两种形式,分别适用于不同版本的\texttt{biblatex}。
  \begin{description}
  \item[3.3+ (20160301)] 由于姓名处理的宏更新,请使用哈希字符串列表指定需要强调
    的作者。
\begin{verbatim}
    \initauthors{{72b3cccfc646adeb1d6b20320b56fd7d}}
\end{verbatim}
  \item[3.2- (20151228)] 使用旧式的姓名处理宏,使用下列形式的作者列表。
\begin{verbatim}
    \forcsvlist{\listadd\boldnames}{{Qi, F\bibinitperiod}}
\end{verbatim}
  \end{description}


  % \forcsvlist{\listadd\boldnames}{{Qi, F\bibinitperiod}}
  \initauthors{{72b3cccfc646adeb1d6b20320b56fd7d}}
  \printbibliography[prefixnumbers=J,heading=cvtype,title={相关工作}]

\end{refsection}

\section{工作条件}
\label{sec:devices-laboratory}

\begin{hcomment}
  (包括已具备的实验条件,尚缺少的实验条件和拟解决的途径,包括利用国家实验室、国
  家重点实验室和部门重点实验室等研究基地的计划与落实情况)
\end{hcomment}

略。

\section{承担科研项目情况}
\label{sec:projects}

\begin{hcomment}
  (申请人和项目组主要参与者正在承担的科研项目情况,包括国家自然科学基金项目,
  要注明项目的名称和编号、经费来源、起止年月、与本项目的关系及负责的内容等)
\end{hcomment}

略。

\section{完成国家自然科学基金项目情况}
\label{sec:prev-nsfc}

\begin{hcomment}
  (对申请人负责的前一个已结题科学基金项目(项目名称及批准号)完成情况、后续研究
  进展及与本申请项目的关系加以详细说明。另附该已结题项目研究工作总结摘要(限500
  字)和相关成果的详细目录)
\end{hcomment}

略。

\part{资金预算说明}
\label{sec:finance}

\begin{hcomment}
  购置单项经费5万元以上固定资产及设备等,须逐项说明与项目研究的直接相关性及必要
  性。
\end{hcomment}

略。

\part{其他需要说明的问题}
\label{sec:others}

略。


\end{document}

%%%%%%%%%%%%%%%%%%%%%%%%%%%%%%%%%%%%%%%%%%%%%%%%%%%%%%%%%%%%%%%%%%%%%%
%%% my-nsfc-proposal.tex ends here
